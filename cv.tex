%%%%%%%%%%%%%%%%%%%%%%%%%%%%%%%%%%%%%%%%%
% Friggeri Resume/CV for A4 paper format
% XeLaTeX Template
% Version 1.1
%
% A4 version author:
% Marvin Frommhold (depressiverobot.com)
% https://github.com/depressiveRobot/friggeri-cv-a4
%
% Original author:
% Adrien Friggeri (adrien@friggeri.net)
% https://github.com/afriggeri/CV
%
% License:
% CC BY-NC-SA 3.0 (http://creativecommons.org/licenses/by-nc-sa/3.0/)
%
% Important notes:
% This template needs to be compiled with XeLaTeX and the bibliography, if used,
% needs to be compiled with biber rather than bibtex.
%
%%%%%%%%%%%%%%%%%%%%%%%%%%%%%%%%%%%%%%%%%

% Options
% 'print': remove colors from this template for printing
% 'nocolors' to disable colors in section headers
\documentclass[]{friggeri-cv-a4}
\renewcommand{\labelitemii}{$\cdot$}

\addbibresource{bibliography.bib} % Specify the bibliography file to include publications

\begin{document}

\header{Honglin }{Zhang} % Your name and current job title/field

%----------------------------------------------------------------------------------------
% SIDEBAR SECTION
%----------------------------------------------------------------------------------------

\begin{aside} % In the aside, each new line forces a line break
\section{about}
15907 SE 48th Dr,
Bellevue, WA
98006
\href{tel:(215) 460-1823}{(215) 460-1823}
\href{mailto:honglinz.t@gmail.com}{honglinz.t@gmail.com}
\section{specialties}
Java/Scala, C/C++, Haskell, Javascript/Typescript, Python/Ruby/Perl, Emacs/VS Code/IntelliJ, Kafka, NoSQL, Mesos/K8S, Distributed Systems, Transpiler/Type Systems, Machine Learning Infra, Scientific Computation
\end{aside}

%----------------------------------------------------------------------------------------
% WORK EXPERIENCE SECTION
%----------------------------------------------------------------------------------------
\section{experience}
\begin{entrylist}
%------------------------------------------------
\entry
    {2022-Now}
    {Twitter, Content Health Signals}
    {Staff Software Engineer, Tech Lead}
    {
      newly founded team, 3 engineer new hires, 0.5 EM, 0.5 PM
      \begin{itemize}
      \item process
        \begin{itemize}
        \item introduced sprint planning process to the team
        \item led the team agreement on ``how we operate as a team" in doc
        \end{itemize}
      \item mentoring
        \begin{itemize}
        \item onboarding new hires to the tech stack at Twitter
        \item mentoring engs on the team on how to mentor
        \end{itemize}
      \item roadmap
        \begin{itemize}
        \item developing the team roadmap for the next 1 year, focusing on incremental customer value and long-term direction alignment
        \end{itemize}
      \end{itemize}
    }
\end{entrylist}
\begin{entrylist}
  \entry
      {2021-2022}
      {Twitter, Content Health Detection Infra}
      {Staff Software Engineer, Tech Lead}
      {
        \begin{itemize}
        \item process
          \begin{itemize}
          \item coached others to streamline sprint planning, customer request triaging, and backlog prioritization
          \end{itemize}
        \item mentoring
          \begin{itemize}
          \item onboarded 1 EMs, 2 PMs, 1 TPM, 1 SRE and 6+ engs from 3 timezones in 6+ locations
          \item help engineers on the team and in other orgs grow and get promoted
          \item identified and grew the new TL on the team
          \item recurring 1:1s with project leads on communication, delegation and technical ambiguity
          \end{itemize}
        \item team execution and KR
          \begin{itemize}
          \item led the execution of merging the acquired rules engine, Smyte, with the on-prem rules engine
          \item broke down multi-year projects to incremental projects, identified project leads and held project leads accountable
          \item established project progress newsletter cadence and handed off to a TPM
          \item developed team quarterly and yearly KRs and coached the new TL to continue driving roadmap development
          \end{itemize}
        \item cross-team collaboration
          \begin{itemize}
          \item lead contributor in defining the Technical Design Review process in the org
          \item lead contributor in estabilishing the process of content health incidents
          \item member of promo commitee for Staff engineer candidates
          \end{itemize}
        \item key event timelines
          \begin{itemize}
          \item transferred the TL responsibility to the new TL in early 2022
          \end{itemize}
        \end{itemize}
      }
\end{entrylist}
\begin{entrylist}
  \entry
      {2017-2020}
      {Twitter, Content Health Detection Infra}
      {Senior Software Engineer}
      {
        \begin{itemize}
        \item process
          \begin{itemize}
          \item introduced the process of sprint planning and customer request triaging to the team
          \end{itemize}
        \item mentoring
          \begin{itemize}
          \item onboarded 1 EMs and 3 engs from acquisition
          \end{itemize}
        \item rules engine improvement
          \begin{itemize}
          \item ran parity checks and led parity fixes for a re-written rules engine
          \item proposed and led a 3-person group to migrate the legacy rules engine to the newly re-written one without downtime
          \item proposed to merge the acquired rules engine, Smyte, with the on-prem rules engine after building a transpiler prototype
          \end{itemize}
        \item cross-team collaboration
          \begin{itemize}
          \item threat modeling with the security team. The result  was used as an example for the company.
          \item piloted with NoSQL DB team to adopt RocksDB for the company and provided feedback to the rollout plan
          \end{itemize}
        \item key event timelines
          \begin{itemize}
          \item acted as the team Tech Lead since late 2018, officially titled as the Tech Lead in early 2020
          \item promoted to Staff Software Engineer in early 2021
          \end{itemize}
        \end{itemize}
      }
\end{entrylist}
\begin{entrylist}
  \entry
  {2015-2017}
  {Twitter, Content Health Detection Infra}
  {Software Engineer II}
  {
    one of the 4 founding members of the only infra team in the org
    \begin{itemize}
    \item process
      \begin{itemize}
      \item proposed, led and advocated oncall best practices to reduce the operational burden on the team. Go-to person for investigating system incidents, build and deploy failures
      \end{itemize}
    \item timeseries DB improvement
      \begin{itemize}
      \item proposed, advocated and built a HyperLogLog timeseries database and integrated with the rules engine, resulting in 60\% to 80\% QPS reduction compared to a regular counter timeseries database
      \item built a Heron-based generic Monoid aggregation pipeline to avoid counter write-path race conditions, adopted by the decayed counter timeseries database
      \end{itemize}
    \item rules engine improvement
      \begin{itemize}
      \item introduced Thrift-based Object-Oriented support to the Domain Specific Language in the rules engine
      \end{itemize}
    \item cross-team collaboration
      \begin{itemize}
      \item built integrations between the on-prem rules engine and the acquired Vine and Periscope products on AWS to expand the surface of content health detection
      \end{itemize}
    \item promoted to Senior Software Engineer in 2017
    \end{itemize}
  }
\end{entrylist}
\begin{entrylist}
\entry
    {2015}
    {Amazon Web Services, CloudFront}
    {Software Development Engineer}
    {
      performance and scalability team
      \begin{itemize}
      \item built a system that checks the whole SSL trust chain and sends notifications before any certificate in the chain expires
      \item converted a gatekeeper health check model to a centraizlied health check model, saving CPU cycles on thousands of machines
      \item simplified the end-customer request flow from 5 hops to 4 hops, providing extra capacity during the peak traffic season
      \end{itemize}
    }
\entry
    {2014}
    {Amazon Web Services, Silk Browser}
    {Software Development Engineer}
    {rendering engine team
      \begin{itemize}
      \item fixed bugs and security issues on WebKit rendering engine (C++, \textasciitilde1,000,000 lines code base)
      \item reduced metrics dashboard loading time from 20+ seconds to 6 seconds with python profiling and DB migration
      \end{itemize}
    }
\entry
    {2013}
    {Amazon Web Services, Security Tools}
    {Software Development Engineer Intern}
    {
      detection and triaging platform team
      \begin{itemize}
      \item algorithmic level webpage rendering time improvement, reducing rendering time from 70+ seconds to 10 seconds
      \item built security monitoring service with Ruby on Rails from scratch using EC2, S3, CloudFormation, CloudSearch, RDS, SNS, SQS, IAM.
      \end{itemize}
    }
%------------------------------------------------
\end{entrylist}
%----------------------------------------------------------------------------------------
% EDUCATION SECTION
%----------------------------------------------------------------------------------------
\section{education}
\begin{entrylist}
%------------------------------------------------
\entry
    {2012--2014}
    {Master of Computer and Information Technology}
    {}
    {University of Pennsylvania, Philadelphia, PA}
%------------------------------------------------
\entry
    {2008--2012}
    {Bachelor of Science, Physics and Mathematics}
    {}
    {Tsinghua University, Beijing, China}
%------------------------------------------------
\end{entrylist}
%----------------------------------------------------------------------------------------
\section{tech talks}
\begin{entrylist}

%------------------------------------------------

\entry
{2018}
{\href{https://www.meetup.com/Seattle-Scala-User-Group/events/tmkmjpyxnbmb/}{Kafka \& Content Health Infra at Twitter}}
{Scala at the Sea}
{100+ participants; recruited 1 engineer; more talks were scheduled at Twitter due to the success of this first talk}
%------------------------------------------------

\end{entrylist}
\end{document}
